\documentclass[12pt]{article}

\begin{document}

\section{Introduction}
\subsection*{Slide 2}
What are the different approaches to clustering?
\begin{itemize}
	\item Cost-based clustering\\
	Find minimum cost partition. $k$-means (minimize sum of squared distances), $k$-median (minimize sum of distances, centers from data), fuzzy $k$-means (aims to minimize the weighted sum of squared distances), $k$-medoids (Partitioning Around Medoids). \\
	All the algorithms are some variants of local search heuristics.
	\item Hierarchical clustering\\
	Start with every point in its own cluster. Merge till one cluster remains. Common ways to merge, single-linkage, max-linkage, avg-linkage.
	\item Spectral clustering\\
	Let $L = D^{-1/2} A D^{-1/2}$ (where $A$ is adjacency matrix). Find top $k$ eigenvalues of this matrix and cluster using $k$-means.\\
	There are relations between these and kernel PCA.
\end{itemize}
Some recent work on cost-based hierarchical clustering. The cost is based on trying to put similar points higher on the tree.

\subsection*{Slide 4}
Computational complexity of clustering.
	\begin{itemize}
		\item $[$Dasgupta' 08$]$\\
		Reduction from NAE3SAT. Hard for $k=2$ and $d = n$.

		\item $[$Vattani' 09$]$\\
		Planar construction. Hard for $d=2$ and $k=n^{\epsilon}$.

		\item $[$Awasthi et. al '15$]$\\
		There exists a constant $\epsilon$ such that it is NP-Hard to approximate $k$-means within $(1+\epsilon)$. Reduction from  vertex cover on triangle free graphs which is NP-Hard to approximate within $1.36$ for $k = \Omega(n)$.

		\item $[$Megiddot and Supowit$]$\\
		The $k$-median problem as well as the $k$-center problem is NP-Hard. They even show that $k$-center is NP-Hard to approximate within a constant factor for $d = 2$ for $k =\Omega(n)$.

		\item $[$Arya et. al '01$]$\\
		The $k$-median problem has a local search based algorithm with approximation ratio $3+\epsilon$.\\
	This has been improved to $1+\sqrt{3}+\epsilon$.
	\end{itemize}

\subsection*{Slide 10-11}
Definition of structure or clusterability conditions.
	\begin{itemize}
		\item $\alpha$-center proximity\\
		$k$-means efficiently solvable under center-proximity for $\alpha > 2 + \sqrt{3}$. NP-Hard for $\alpha \le 3$.\\
		 $k$-median efficiently solvable under center-proximity for $\alpha > 1 + \sqrt{2}$. NP-Hard for $\alpha \le 2$
		 
		 \item $(\alpha, \epsilon)$-center proximity\\
		 Efficient $(1+\frac{8n}{m(C)}\epsilon)$-approximation to the $k$-median objective for $\alpha > 2 + \sqrt{7}$.
		 
		 \item $\epsilon$-separatedness\\
		 $Cost(OPT(k)) \le \epsilon^2 Cost(OPT(k-1))$. Variant of $k$-means finds a $(1+\epsilon)$-approximation with high probability in time $O(nk+k^3)$. \\
		 For $k=2$, algorithm succeeds with probability $1-\frac{20\epsilon^2}{\sqrt{(1-\epsilon^2)(1-101\epsilon^2)}}$. This implies average radius much smaller than separation.
		 $r_i < \frac{\epsilon}{\sqrt{1-\epsilon^2}} d(c_i, c_j)$.  
	\end{itemize}

\subsection*{Slide 14}
\subsection*{Slide 22}
\subsection*{Slide 24}

\end{document}
